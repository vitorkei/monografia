%% ---------------------------------------------------------------------------- %
\chapter*{Resumo}
%% ---------------------------------------------------------------------------- %
%
\noindent%
TAMADA, V. K. T. \textbf{Estudo de caso de Deep Q-Learning}. Trabalho de Conclusão de Curso
 - Instituto de Matemática e Estatística, Universidade de São Paulo,
São Paulo, 2018.
\\

\textit{Deep Q-Learning} é uma técnica de aprendizado de máquina que une a capacidade de visualização de imagem e detecção de características de redes neurais convolucionais com aprendizado por reforço para ensinar uma inteligência artificial a ter sucesso no ambiente em que se encontra.
Por conta disso, um agente que aprende por esse método se assemelha a forma como pessoas aprendem a, por exemplo, jogar um jogo quando não utilizam manual de instruções ou recebem ajuda externa.

O objetivo do trabalho foi avaliar a capacidade de uma inteligência artificial aprender a ter sucesso em um ambiente utilizando essa técnica de aprendizado de máquina e as dificuldades do processo.
Com isso, estudar uma técnica de aprendizado de máquina que não é ensinada nas matérias de inteligência artificial da graduação.
A implementação foi feita utilizando TensorFlow na linguagem Python.

Os experimentos foram feitos em três ambientes de características e graus de dificuldades distintos, para verificar a flexibilidade do \textit{deep Q-learning} tanto dos resultados quanto da dificuldade de se obtê-los.
Os ambientes foram o \textit{Gridworld}, \textit{Pong} do Atari2600, e \textit{Asteroids} do Atari2600.
Enquanto o primeiro é o ambiente mais simples, podendo ser resolvidos com técnicas menos complexas, os jogos do Atari2600 precisaram utilizar \textit{Deep Q-Learning} por utilizar a tela do jogo como entrada, assim como uma pessoa faria.
O desempenho foi avaliado pelo sucesso de se alcançar o objetivo, no caso do \textit{Gridworld}, e pela pontuação obtida, no caso do \textit{Pong} e do \textit{Asteroids}.

A implementação obteve sucesso no \textit{Gridworld} e no \textit{Pong}, mas não no \textit{Asteroids}.
A arquitetura da rede e número de episódios de treinamento foi diferente para cada ambiente.
Trabalhos relacionados indicam que \textit{Asteroids} obteria sucesso com as devidas otimizações, escolhas de hiper-parâmetros e suficiente tempo de treinamento.

\\

%Visualização de imagem, abstração de informação e aprendizado por recompensa são tarefas que seres humanos aprendem consideravelmente rápido.
%Computadores, por outro lado, podem levar horas ou até dias para aprender algo que pessoas fariam em segundos, principalmente quando envolve interpretação de imagens.

%Este trabalho buscou estudar a eficiência e capacidade de um agente aprender utilizando \textit{deep Q-learning}.
%Aplicando a técnica em três ambientes de características e complexidade diferentes, \textit{Gridworld}, \textit{Pong} e \textit{Asteroids}, foi analisado o grau de sucesso ou insucesso que a inteligência artificial teve neles.
%No fim, ao observar resultados correspondentes ao nível de dificuldade de cada domínio, é feita uma discussão sobre como suas diferenças foram relevantes para esses níveis de aprendizado.
%\\

%Este trabalho aplicou a técnica de aprendizado de máquina \textit{deep Q-learning} em três ambientes de características e níveis de complexidade distintos para avaliar sua eficiência e capacidade de criar modelos para um agente se comportar.
%Este trabalho estudou a eficiência do uso de \textit{deep Q-learning} para o aprendizado de um agente em três ambientes de características e níveis de complexidade distintos.

%Utilizando uma união de aprendizado profundo e aprendizado por reforço, o \textit{deep Q-learning}, este trabalho busca estudar a eficiência de \textit{deep Q-learning} para o aprendizado de um agente que recebe apenas a tela de ambientes como entrada em três ambientes diferentes.


%Jogos eletrônicos se tornaram comuns na vida de muitas pessoas nos últimos anos, seja em consoles de mesa tradicionais, em portáteis, ou em celulares.
%Eles normalmente são simples e intuitivos, para qualquer um poder começar a qualquer momento e aprender rapidamente como se joga.
%Porém, isso não é uma tarefa tão fácil para computadores.
%Utilizando \textit{deep learning} em conjunto com aprendizado por reforço, o objetivo deste trabalho é produzir uma inteligência artificial que aprenda a jogar o jogo de Atari2600 \textit{Asteroids}.

\noindent%
\textbf{Palavras-chave:} inteligência artificial, deep q-learning, estudo de caso

