% labels:
% cap:conclusoes

% ---------------------------------------------------------------------------- %
\chapter{Conclusão}
\label{cap:conclusoes}
% ---------------------------------------------------------------------------- %

Motivado pelo interesse em uma técnica de aprendizado de máquina não visto nas disciplinas de inteligência artificial da graduação, este trabalho buscou conhecer, estudar e explorar uma das formas utilizadas para ensinar um agente a se comportar em um domínio utilizando apenas imagens como entrada.
Os ambientes de características e graus de complexidade distintos permitiram avaliar as capacidades e dificuldades que essa técnica apresenta, com resultados que refletiram as expectativas ainda que apenas em parte.
Mesmo a falta de sucesso serviu como elemento de análise para este estudo.

O \textbf{\textit{Gridworld}} apresentou sucesso consistente em encontrar um caminho até o objetivo.
Isto estava dentro do esperado por conta da baixa dimensionalidade do problema:
existem poucos estados, poucas ações disponíveis e não há aleatoriedade.
Há poucas informações para o agente aprender, podendo inclusive ser resolvido por técnicas mais simples sem grandes problemas.

O \textbf{\textit{Pong}} já se mostrou mais complicado.
Ainda que seu aprendizado tenha sido \hyperref[fig:pong_score]{promissor} para os hiper-parâmetros utilizados, ele foi lento e a arquitetura é bem sensível, podendo não conseguir aprender por causa de pequenas alterações.
Por possuir um espaço de estados bem maior que o \textit{Gridworld}, mas bem menos situações diferentes e ações disponíveis que o \textit{Asteroids}, esse resultado refletiu o grau de complexidade médio do ambiente neste trabalho: possível, mas precisa ser feito com cuidado.
Um fato interessante é que, apesar de ainda haver espaço para o agente aprender, como é possível perceber no \hyperref[fig:pong_score]{gráfico de pontuação final do agente}, ele obteve 21 pontos contra a IA nativa do jogo quando o modelo foi colocado a prova.

Por fim, o \textbf{\textit{Asteroids}} não apresentou resultados promissores.
Sendo o ambiente mais complexo dos três, com mais regras e ações para se aprender e uma variedade maior de estados possíveis, o agente não conseguiu criar um modelo que conseguisse uma pontuação boa com as arquiteturas testadas.
Existem diversas técnicas que aceleram o aprendizado de \textit{deep Q-learning} e que poderiam ser aplicadas neste ambiente para tentar obter resultados positivos.
Entretanto, isso seria uma garantia somente se um conjunto de hiper-parâmetros e funções boas fosse utilizado, o que já é um obstáculo por si só considerando a quantidade que existe para serem ajustados e o tempo consumido pelos treinamentos.

\textit{Deep Q-learning} apresentou resultados positivos em comparação com o que se esperava.
Sua dificuldade de uso e tempo consumido são compensados pela capacidade de resolver problemas complexos com pouca influência do desenvolvedor.

%Aprendizado de máquina é uma área com amplo espaço para desenvolvimento.
%Os avanços teóricos dos últimos anos, impulsionados por melhorias nas partes físicas dos computadores permitiram i

%Este trabalho permitiu conhecer e explorar uma técnica de aprendizado de máquina que não é discutida nas disciplinas da graduação.
%A sua capacidade de resolver problemas complexos mostrou-se compensada pela dificuldade de se utilizar com sucesso.
%
%As observações feitas refletiram as expectativas mesmo que apenas em parte.
%Todos utilizaram uma matriz como entrada, mas graus diferentes de dificuldade para se aprender, como o aumento do tamanho da entrada, do espaço de estados e das ações disponíveis.
%O ambiente mais simples, com poucos estados e com recompensa e penalidades bem definidas obteve o maior grau de sucesso;
%o de média complexidade apresentou resultados promissores, mas sem sucessos consistentes como no anterior;
%e o mais complexo, contrariando as expectativas iniciais, foi pouco promissor.


%---------------%

%Este trabalho permitiu explorar uma técnica de aprendizagem de máquina que não é discutida nas disciplinas da graduação, ainda que seja uma junção de duas que são abordadas.
%A complexidade dos problemas que ela é capaz de resolver foi balanceada pela dificuldade de se utilizar com sucesso no ambiente almejado por este projeto.
%Há muitos hiper-parâmetros para se ajustar e o treinamento leva horas, até mesmo dias, para terminar.
%Além disso, não existem regras e teorias bem definidas para quais valores devem ser adotados, apenas relatos de casos bem sucedidos e algumas direções baseadas neles de quais podem ser bons.
%Mesmo assim, seu estudo e desenvolvimento foram interessantes, ainda que sucesso com o \textit{Asteroids} não tenha sido obtido no final por conta de sua complexidade.



%A escolha do tema não foi muito difícil. Já estava seguindo a trilha de inteligência artificial do curso de ciência da computação e o grande interesse por jogos mostrou rapidamente uma aplicação do que foi estudado nessa área.
%A implementação também não foi muito problemática, uma vez que o estudo e desenvolvimento de redes neurais profundas e respectivas aplicações para detecção de imagem tiveram amplo destaque nos últimos anos.
%Isso fez muitos guias, escritos ou em vídeo, surgirem pela internet, assim como dúvidas dos mais variados tipos em fóruns do assunto, que serviram de grande ajuda para escrever o código.

%Entretanto, o maior obstáculo teve que ser solucionado por conta própria: encontrar os hiper-parâmetros e funções certos para que o agente aprendesse.
%Existem muitos hiper-parâmetros para ajustar e o treinamento leva horas, podendo até mesmo passar de um dia para o outro ou até além disso para terminar.
%Também não existem regras e teorias bem definidas para a escolha de muitos deles, apenas relatos de alguns que foram usados e funcionaram, o que dá uma noção de valores bons no máximo.
%Os testes com ambientes mais simples serviram de grande ajuda ao garantir e demonstrar o funcionamento da técnica de aprendizado \textit{deep Q-learning}, indicando que a dificuldade estava na escolha correta dos hiper-parâmetros pelo fato de \textit{Asteroids} ser um ambiente muito mais complexo.
